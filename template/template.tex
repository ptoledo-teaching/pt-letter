%%
%% PT Letter Template - Parametrized Version
%% Using parametrized commands for sender and recipient
%%
%% Author: Pedro Toledo Correa
%% Version: 0.1
%% Date: 2025-10-18
%%

\documentclass[spanish]{pt-letter}

% Sender information
\ptsendername{Luke Skywalker}
\ptsenderaffiliation{Jedi Knight\\Rebel Alliance}
\ptsenderaddress{Yavin IV Base, Outer Rim Territories}
\ptsenderemail{luke.skywalker@rebelalliance.org}
\ptsenderphone{+999 (0)4 ABY-JEDI}
\ptsendersignature{figures/signature.png}

% Recipient information
\ptrecipientname{Obi-Wan Kenobi}
\ptrecipientaffiliation{Jedi Master\\Former General of the Grand Army of the Republic}
\ptrecipientaddress{Tatooine Desert, Dune Sea Region}
\ptrecipientemail{ben.kenobi@jediorder.org}
\ptrecipientphone{+999 (0)19 BBY-HIDE}

% Letter location and date
\ptletterlocation{Yavin IV}
\ptletterdate{4 ABY}

\begin{document}

\opening{Estimado Maestro Kenobi,}

Me dirijo a usted con gran entusiasmo para presentarle la \textbf{Suite PT de Templates LaTeX}, un conjunto de herramientas documentales que he desarrollado para facilitar la creación de documentos profesionales en la galaxia académica.

Como bien sabe, durante mi entrenamiento Jedi aprendí que la claridad y la precisión son fundamentales, no solo en el manejo de la Fuerza, sino también en la comunicación escrita. Es por ello que he creado esta colección de plantillas que buscan estandarizar y mejorar la calidad de nuestros documentos.

\subsection*{Componentes de la Suite PT}

La suite está compuesta por los siguientes módulos:

\begin{center}
    \begin{tblr}{colspec={|l|c|c|}}
        \hline
        \tableheader
        Componente              & Versión & Estado                              \\
        \hline
        \hline
        \inlinecode{pt-commons} & v0.1    & \textcolor{ptgreen}{Activo}         \\
        \hline
        \inlinecode{pt-article} & v0.1    & \textcolor{ptgreen}{Activo}         \\
        \hline
        \inlinecode{pt-slides}  & v0.1    & \textcolor{ptgreen}{Activo}         \\
        \hline
        \inlinecode{pt-letter}  & v0.1    & \textcolor{ptgreen}{Activo}         \\
        \hline
        \inlinecode{pt-report}  & v0.1    & \textcolor{ptyellow}{En desarrollo} \\
        \hline
        \inlinecode{pt-book}    & v0.1    & \textcolor{ptyellow}{En desarrollo} \\
        \hline
    \end{tblr}
\end{center}

\subsection*{Características Principales}

La suite ofrece las siguientes funcionalidades:

\begin{itemize}
    \item \textbf{Estandarización:} Todos los componentes comparten un estilo visual coherente
    \item \textbf{Multiidioma:} Soporte para español, inglés, portugués y francés
    \item \textbf{Tablas Avanzadas:} Integración con \inlinecode{tabularray} para tablas profesionales
    \item \textbf{Código Fuente:} Resaltado de sintaxis con \inlinecode{minted}
    \item \textbf{Matemáticas:} Soporte completo para ecuaciones con \inlinecode{amsmath}
    \item \textbf{Gráficos:} Integración con \inlinecode{tikz} y \inlinecode{pgfplots}
    \item \textbf{Control de Versiones:} Sistema integrado de versionado de documentos
\end{itemize}

Como puede apreciar en esta misma carta, el módulo \inlinecode{pt-letter} permite crear correspondencia formal con un diseño elegante y profesional, utilizando comandos parametrizados que simplifican enormemente el proceso.

Maestro, creo que estas herramientas pueden ser de gran utilidad para la comunidad académica galáctica. Su sabiduría y experiencia serían invaluables para mejorar y expandir este proyecto.

Quedo a su disposición para cualquier consulta o sugerencia que desee compartir. Que la Fuerza lo acompañe en sus proyectos documentales.

Atte,

\end{document}
